\documentclass[12pt]{article}
%\documentclass[10pt,letterpaper]{article}
\usepackage[letterpaper]{geometry}
\usepackage[utf8]{inputenc}
\usepackage[english]{babel}
\usepackage{graphicx}
\usepackage{cancel}
\usepackage{amsmath}
\usepackage{ulem}
\usepackage{authblk}
\usepackage{indentfirst}
\usepackage{enumitem}
\usepackage{cancel}
\usepackage{soul}
\newenvironment{QandA}{\begin{enumerate}[label=\bfseries\arabic*.]\bfseries}
                      {\end{enumerate}}
\newenvironment{answered}{\par\quad\normalfont}{}

\title{CIS501: Out-of-Order \\ (Dynamic Scheduling)}
\author[1]{Shreyas S. Shivakumar}

\begin{document}

\maketitle

\section{Lecture Notes}

\begin{QandA}
\item Why do we need \textit{code scheduling} and what are the \textit{two ways} of doing it?
\begin{answered}
\vspace{-0.85cm}
\begin{itemize}
    \item We can't really avoid \textbf{load-to-use} delays in the designs that we have looked at so far. But what if we could \textbf{fill} those delay slots with instructions that are \textit{independent} of it? This way we can \textit{increase CPI} and \textit{increase ILP}. 
    \item \textbf{Static Scheduling:} at \textit{compile} time, done by software. 
    \item \textbf{Dynamic Scheduling:} at \textit{run} time, done by hardware.
    \item We will be primarily looking at \textit{dynamic scheduling}; a process that creates more opportunities to schedule things in parallel by moving independent instructions around. This is very common today and exists on cellphones, laptops, etc.
\end{itemize}
\end{answered}

\item Static Scheduling
%\vspace{-0.85cm}
\begin{answered}
Some key characteristics / criteria to properly perform \textit{static scheduling} are listed below. These must be taken care of at \textit{compile time}. 
\begin{itemize}
    \item \textbf{Scheduling Scope:} Ideally we want a larger scope because if the scope is too small there's nothing that the compiler can do to improve performance. It cannot rearrange and re-order instructions if it doesn't see enough instructions. \textit{Branch Instructions} specifically can cause problems with scheduling scope because it is impossible to move memory operations past branch instructions. 
    \item \textbf{Registers:} We need a sufficient amount of registers to hold all \textit{"live"} values. But we are limited by the ISA in how many registers are available to us. 
    \item \textbf{Alias Analysis:} We need to be able to tell whether a LOAD/STORE instruction reference the \textit{same memory locations}. This is often difficult for the compiler to do and prevents reordering of loads above stores. 
\end{itemize}

However, there are situations where \textit{Static Scheduling} works really well such as in \textbf{SAXPY} (Single-precision AX Plus Y). Since all loop iterations are independent - loop unrolling. Alias analysis is tractable. Limited registers are still a challenge. 
\end{answered}

\item Dynamic Scheduling
\begin{answered}
In this variant, the \textit{hardware} is responsible for the re-scheduling of instructions. The key motivations and criteria are similar to the ones mentioned above. We need to increase \textit{scheduling scope} and we can use \textit{branch prediction} to unroll branches. The benefit of this approach is that everything happens under the hood and the software layers are not exposed to any of the inner mechanisms. 

The \textbf{Out-of-Order Pipeline} can be divided into the following modules
\begin{itemize}
    \item \textbf{In-Order} Front End - Fetch, Decode, Rename, Dispatch
    \item \textbf{Out-of-Order} Execution - Buffer of Instruction $\rightarrow$ \{ Issue, Register Read, Execute, Write-back \} 
    \item \textbf{In-Order} Commit - Commit 
\end{itemize}

Programs will enter \textit{in-order} and exit \textit{in-order}, maintaining the illusion of sequential program execution. But between the \textit{Dispatch} and \textit{Commit} stages, the instructions can be processed out of order and hence we can reap the benefits of an out-of-order design while maintaining a final sequential ordering. 

\textbf{Dependency Types:}
\begin{itemize}
    \item \textbf{RAW} (Read After Write) - \textit{True Dependence}
    \item \textbf{WAW} (Write After Write) - \textit{False Dependence}
    \item \textbf{WAR} (Write After Read) - \textit{False Dependence}
\end{itemize}

Both \textbf{WAW} and \textbf{WAR} are \textit{False} dependencies and can be eliminated completely by \textit{Register Renaming}. The \textit{True} dependency \textbf{RAW} is enforced by the final \textit{Dynamic Scheduling}. 
\end{answered}

\item Part 1: Register Renaming
\begin{answered}
To understand the \textit{Register Renaming} algorithm, the following new ideas are required:

\textit{Q: What is the difference between architectural registers and physical registers?}

\quad In an ISA such as LC4, we have registers $R0...R8$. These are actually indirections / pointers to actual physical registers which are more than 8 in number. The pointers to the physical register locations are called \textit{architectural registers} and the actual physical locations are \textit{physical registers}.
\begin{equation*}
    \#\text{physical registers} > \#\text{architectural registers}
\end{equation*}

\textit{Q: What are the additional hardware structures required for register renaming?}

\quad The first is a \textbf{Map Table}, which maintains the mapping from \textit{architectural registers} to the \textit{physical registers}. The second is a \textbf{Free List}, which maintains the list of available registers (in a queue). 

\textit{Q: What is the goal of the Register Renaming algorithm?}

\quad The goal is to \textit{eliminate contention} / \textit{eliminate false dependencies}. 
\end{answered}

\item Part 2: Dynamic Scheduling
\begin{answered}
The previous \textit{register renaming} happened in-order, and the instructions then move onto the actual dynamic scheduling, i.e when the instructions no longer follow the program order. By this stage, all the \textit{False Dependencies} have been removed and only \textit{True Dependencies} remain. 

The instructions from the previous stage are put into a \textbf{Instruction Buffer} or \textit{Instruction Window} / \textit{Instruction Scheduler}. 

Additionally, there is another hardware structure called the \textbf{Ready Table} which maintains whether an instruction is \textit{ready} or not, and then executes the oldest \textit{ready} instruction. The ready-ness of the instruction is maintained by a single \textbf{Ready-Bit}. 

\textit{Q: Why give preference to the oldest instruction?}

\quad Old instructions will produce things that more instructions may depend on. By processing older instructions first we also guarantee that \textit{progress} will be made. 

The other key pieces of hardware are the \textbf{Issue Queue}, which is the \textit{central piece} of scheduling logic. It holds the instructions from the \textit{Dispatch Stage} to the \textit{Issue Stage}. It tracks \text{Ready} inputs as well as their \textit{Birthday} to indicate how old each instruction is. 

\textit{Q: Why maintain birthday instead of age?}

\quad Age changes over time but birthday remains consistent and is hence easier to maintain. 

\textit{Note:} The Issue Queue is not \textit{read} in program order which means that the structure itself contains instructions not in program order. This is technically not a \textit{queue} by the proper definition and works more like an Issue Table. 

The next piece of hardware is the \textbf{Re-order Buffer (ROB)}, which holds the instructions from the \textit{Fetch Stage} through to the \textit{Commit Stage}. 

In summary:
\begin{itemize}
    \item The \textit{Dispatch Stage} fills up the \textit{Issue Queue} and the \textit{Issue Stage} involves figuring out which instructions in the \textit{Issue Queue} are ready and can leave the queue to go onto the next stage.
    \item This involves \textbf{Selecting} ready instructions that can leave and \textbf{Waking Up} dependent instructions. This \textit{search} and \textit{wake-up} is done using Content Addressable Memory (CAM).
\end{itemize}  
\end{answered}

\item Register Renaming - Continued
\begin{answered}
In the previous description of the \textit{register renaming} algorithm, once a register was overwritten it was \textit{discarded}. But in reality, there is a finite number of these registers and they need to be \textbf{reused}. We can do this by \textit{tracking / logging} the overwritten registers in the \textbf{Re-Order Buffer} (ROB) and then we can \textbf{free} them after the \textbf{Commit} stage. 

\quad The only change to the algorithm is that we need to keep track of the overwritten register before updating the \textit{MapTable} with the new register. Once the instruction is finished at the \textit{Commit Stage}, we can \textbf{free} that register. 

\textbf{Re-Order Buffer (ROB):}

\begin{itemize}
    \item Holds all the information for \textit{recovery} and \textit{commit}.
    \item This information is not removed until the very last \textit{commit}.
    \item Helps track instructions for an in-order \textit{commit}.
    \item Useful in freeing up physical registers.
    \item Useful in recovering from mis-predictions. 
\end{itemize}

\textbf{Recovery:}

Everything that leaves the \textit{Commit Stage} \textbf{must be} correct. This implies that up-to the \textit{Commit Stage}, we are free to use any method of speculation and guessing, but once an instruction leaves the \textit{Commit Stage}, it affects the system permanently. This also gives us the opportunity to \textbf{recover} from mis-speculations prior to the \textit{Commit Stage}. This involves completely removing the wrong path instructions, i.e rewinding all the changed state in the system. This can be done the following ways:
\begin{itemize}
    \item Log-based Reverse Renaming - \textit{track old mapping; slow}
    \item Checkpoint-based Recovery - \textit{checkpoint MapTable and free list; more state}
    \item Hybrid - \textit{combination of both}
\end{itemize}

\textit{Note:} At the \textit{Commit Stage} the instruction becomes \textbf{Architected State}. This means that we've really made the changes and instruction has really happened. 

\textit{Q: Why is it safe to free an overwritten register?}

\quad It is safe because any instruction that requires it's value should read the new updated value and not the value that was overwritten. 
\end{answered}

\item Dynamic Scheduling - Continued
\begin{answered}
To recap, the \textit{Issue Queue} holds all waiting (un-executed) instructions. The \textit{Issue Queue} also holds ready/not-ready status of all the source registers for each instruction. 

\textit{Note:} The \textbf{Di} stage is a combination of \textit{Decode, Rename} and \textit{Dispatch}. 

The different hardware structures necessary to make this function smoothly are:
\begin{itemize}
    \item Map Table
    \item Ready Table
    \item Re-Order Buffer
    \item Issue Queue
\end{itemize}

Once a register is overwritten, it is placed in the \textbf{Re-Order Buffer} to indicate that it can be released / free-d once the instruction is complete, i.e once the instruction leaves the \textit{Commit Stage}. 

\textit{Q: Where do instructions wait?}

\quad In \textbf{In-Order} execution, instructions usually waited in the \textit{Decode Stage}. In \textbf{Out-of-Order} execution, they will need to wait too, but they can wait in more than one place such as the \textbf{Re-Order Buffer} and the \textbf{Issue Queue}. Instructions that are stalled after the \textit{Di Stage} are usually \textit{hanging out} in the \textbf{Issue Queue}.
\end{answered}


\end{QandA}

\end{document}